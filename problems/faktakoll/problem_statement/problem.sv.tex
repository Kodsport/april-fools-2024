\problemname{Faktakollen}
\noindent
Efter populariseringar av stora språkmodeller är det svårare än någonsin att skilja
fakta från lögn. I Python-kommiténs pågående mål att vara en av de bättre språken har
de bestämt sig för att skapa modulen factcheck. Denna erbjuder funktionen istrue:
givet en sträng som beskriver ett påstende, så returnerar den True om det stämmer,
annars False. På något konstigt vis är det nu ditt uppdrag att implementera denna.


\section*{Indata}
\noindent
Indatan består av en enda rad, vilket innehåller ett påstående.

Som tur är så är testdatan för det här problemet snällt. Exempelvis kommer inte paradoxer
såsom "detta påstende är falskt", påstenden vi ej vet sanningsvärdet av, såsom "varje
jämnt naturligt tal större än 2 kan uttryckas som summan av 2 primtal", eller påstenden
som är subjektiva som "Sven är den bästa isbjörnen".

\section*{Utdata}
\noindent
Skriv ut "True" om påståendet är sant, annars "False".

\section*{Poängsättning}
Din lösning kommer att testas på flera testfall.
\noindent
För att lösa problemet måste du klara alla testfall.

\problemname{Gör Kattis glad}

Kattis är inte glad. Kan du få Kattis att bli glad?

\begin{centering}
  \begin{figure}[h]
      \centering
      \includegraphics[width=0.8\textwidth]{tired-kattis.png}
      \caption{Bild på Kattis. Kattis är trött och ledsen över att du lämnar in så många WA. }
  \end{figure}
\end{centering}


\section*{Interaktion}
Du har 10 olika handlingar som du kan göra, i valfri ordning. 

\begin{enumerate}
  \item Hälla 10 ml mjölk precis framför Kattis.
  \item Klappa på Kattis (Mjukt)
  \item Klappa på Kattis (Hårt)
  \item Säg 
\end{enumerate}

Du får göra upp till 20 handlingar efter varandra.
För att göra en handling, skriv ut handlingens nummer på en egen rad. För att klappa på Kattis mjukt, 
och sedan hälla 10 ml mjölk precis framför Kattis, kan du skriva \texttt{2} på en rad, och sedan \texttt{1} på en annan rad.
Se till att inte skriva för mycket på en rad, du vill inte göra Kattis ännu mer ledsen. 

Dina handlingar kommer att utföras en efter en. Ifall du lyckas göra Kattis ännu mer upprörd, 
avslutas ditt program och du får \texttt{Wrong Answer} direkt. 
Du får även \texttt{Wrong Answer} ifall du skriver något som inte är något av de ovannämnda 10 alternativen.

När Kattis blir glad, kommer dina övriga att avbrytas.

\textbf{Se till att flusha outputen efter varje handling}, annars kan du få \textit{Time Limit Exceeded}.
I C++ kan detta göras med exempelvis \texttt{cout << flush;}
eller \texttt{fflush(stdout);},
i Python med \texttt{stdout.flush()}
och i Java med \texttt{System.out.flush();}.

\section*{Poängsättning}
Din lösning kommer att testas på en mängd testfallsgrupper.
För att få poäng för en grupp så måste du klara alla testfall i gruppen.

\noindent
\begin{tabular}{| l | l | l |}
  \hline
  \textbf{Grupp} & \textbf{Poäng} & \textbf{Gränser} \\ \hline
  $1$   & $100$        & Gör Kattis glad. \\ \hline
\end{tabular}

\problemname{Fact Check}
\noindent

After the popularization of large language models, it is harder than ever to distinguish
truth from falsehoods. In their ongoing quest to be one of the better languages,
the Python-commitee has decided to create the module factcheck. This module defines
the function istrue: given a string describing a statement, it returns True if the
statement is True, otherwise False. Somehow, it has now become your task to implement
this functionality.

\section*{Input}
\noindent
The input contains a single string, describing a statement.

Fortunately, the test data for this problem is not evil. For example, no paradoxes such as
"this statement is false", statements we do not know the truthhood of, such as "every even
natural number greater than 2 kan be written as the sum of 2 primes", or subjective statements
such as "Sven is the best polar bear".

\section*{Output}
\noindent
Print "True" if the statement is true, otherwise "False".

\section*{Points}
If your solution correctly answers all questions, you will receive 110 points.

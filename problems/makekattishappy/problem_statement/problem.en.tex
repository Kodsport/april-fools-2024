\problemname{Make Kattis Happy}


\section*{Interaction}
First, you get information about the room you are currently in. 
This will be given as a binary string of length five, where every character is ``\texttt{1}'' if you can go in a certain direction, and ``\texttt{0}'' otherwise.
In order, the characters tell you if you can go up, right, down to the right, down to the left, and left.
So the string \texttt{01001} means that you can go to the right and to the left, but in no other directions.

Then, you choose either to go to an adjacent room, or to use Campus Maps.
To go to an adjacent room you write any of the strings 
``\texttt{up}'', ``\texttt{right}'', ``\texttt{downright}'', ``\texttt{downleft}'' or ``\texttt{left}''.
Then you get information about the new room you are in, according to the same format as above.

To use the app you write ``\texttt{app}''.
The app will then calculate the length of the shortest path from where you are to the goal that uses as few staircases as possible.
Since the app is very sensitive to Integer Overflows, it will crash if this length exceeds $10^9$.
In that case, the app will write $-1$.
Otherwise, the length of the path to the goal is written.

When you reach your destination, you should print out ``\texttt{here}'', and your program should be terminated.


\textbf{Make sure to flush the output after each query}, otherwise you can get \textit{Time Limit Exceeded}.
In C++ this can be done using \texttt{cout << flush;}
or \texttt{fflush(stdout);}
in Python with \texttt{stdout.flush()},
and in Java with \texttt{System.out.flush();}.




\noindent
\begin{tabular}{| l | l | l |}
  \hline
  \textbf{Group} & \textbf{Point value} & \textbf{Constraints} \\ \hline
  $1$   & $100$        & Make Kattis happy. \\ \hline
\end{tabular}
